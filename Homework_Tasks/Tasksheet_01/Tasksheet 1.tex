\documentclass{article}
\usepackage{fancyhdr}
\usepackage[utf8]{inputenc}
\usepackage[english]{babel}
\usepackage{tikz, multicol, graphicx, etoolbox, enumerate, setspace, relsize, mathrsfs, verbatim}
\usepackage{amsmath, amsfonts, amssymb, amsthm, epsfig, epstopdf, titling, url, array, esvect, tikz-3dplot}
\usepackage{graphicx}
\usepackage{hyperref}
\usepackage{xcolor}
\usepackage{pgfplots}
\usepackage{tcolorbox}
\usepackage{amsthm}
\usepackage{cancel}
\usepackage[left=1in,right=1in,top=1in,bottom=1in]{geometry}
\usepackage[tableaux]{prooftrees}

\pagestyle{fancy}
\fancyhf{}
\fancyhead[L,RO]{Tasksheet 1}
\fancyhead[R,RO]{Fundamentals of Computational Mathematics}
\fancyfoot[L,RO]{Xiang Gao}
\fancyfoot[R,RO]{Math 4610}
\renewcommand{\headrulewidth}{0.4pt}% Default \headrulewidth is 0.4pt
\renewcommand{\footrulewidth}{0.4pt}% Default \footrulewidth is 0pt
\def\checkmark{\tikz\fill[scale=0.4](0,.35) -- (.25,0) -- (1,.7) -- (.25,.15) -- cycle;} 

\begin{document}

\section{Task 1}
I have met with the you via zoom on the first day of the class, and the here is a screenshot of the email chain.
\begin{center}
\includegraphics[scale = 0.4]{email.png}
\end{center}

\vspace{5pt}

\section{Task 2}
Here's the \href{https://gobymark.github.io/math4610/}{link}, it has also been sent via email.

\vspace{5pt}

\section{Task 3}

I will be using the {\bf command window on MacOS}, since it comes natively. This information has also been sent via email.

\pagebreak

\section{Task 4}
This has been done, and here is a screenshot of my terminal. After some research, git hub just doesn't want me to clone the repository.
\begin{center}
\includegraphics[scale = 0.4]{terminal.png}
\end{center}


\vspace{5pt}

\section{Task 5}

I first found the definition of Version Control Systems (VCS) from this \href{https://www.geeksforgeeks.org/version-control-systems/}{website}\footnote{\url{https://www.geeksforgeeks.org/version-control-systems/}}. To my understanding, VCS is a huge step-up from a website called Overleaf. They both originated from the same purpose of collaborative coding, that is, if you accept {\LaTeX} as a form of coding.\\
However, with VCS, the team members are able to see each and every changes made to the script itself, which in turns, 
\begin{enumerate}
\item[1.] Makes the collaboration process much more efficient; 
\item[2.] Reduces potential errors and makes the debugging process more enjoyable with its traceability.
\end{enumerate}
Overall, even though I have never tried out any VCS myself, I'm happy to get the exposure I need in this course and be flute in implementing it.  


\end{document}