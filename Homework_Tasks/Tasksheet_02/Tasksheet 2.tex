\documentclass{article}
\usepackage{fancyhdr}
\usepackage[utf8]{inputenc}
\usepackage[english]{babel}
\usepackage{tikz, multicol, graphicx, etoolbox, enumerate, setspace, relsize, mathrsfs, verbatim}
\usepackage{amsmath, amsfonts, amssymb, amsthm, epsfig, epstopdf, titling, url, array, esvect, tikz-3dplot}
\usepackage{graphicx}
\usepackage{hyperref}

\usepackage{listings}
\usepackage{xcolor}

\definecolor{codegreen}{rgb}{0,0.6,0}
\definecolor{codegray}{rgb}{0.5,0.5,0.5}
\definecolor{codepurple}{rgb}{0.58,0,0.82}
\definecolor{backcolour}{rgb}{0.95,0.95,0.92}


\usepackage{pgfplots}
\usepackage{tcolorbox}
\usepackage{amsthm}
\usepackage{cancel}
\usepackage[left=1in,right=1in,top=1in,bottom=1in]{geometry}
\usepackage[tableaux]{prooftrees}

\lstdefinestyle{mystyle}{
    backgroundcolor=\color{backcolour},   
    commentstyle=\color{codegreen},
    keywordstyle=\color{magenta},
    numberstyle=\tiny\color{codegray},
    stringstyle=\color{codepurple},
    basicstyle=\ttfamily\footnotesize,
    breakatwhitespace=false,         
    breaklines=true,                 
    captionpos=b,                    
    keepspaces=true,                 
    numbers=left,                    
    numbersep=5pt,                  
    showspaces=false,                
    showstringspaces=false,
    showtabs=false,                  
    tabsize=2
}

\lstset{style=mystyle}

\pagestyle{fancy}
\fancyhf{}
\fancyhead[L,RO]{Tasksheet 2}
\fancyhead[R,RO]{Fundamentals of Computational Mathematics}
\fancyfoot[L,RO]{Xiang Gao}
\fancyfoot[R,RO]{Math 4610}
\renewcommand{\headrulewidth}{0.4pt}% Default \headrulewidth is 0.4pt
\renewcommand{\footrulewidth}{0.4pt}% Default \footrulewidth is 0pt
\def\checkmark{\tikz\fill[scale=0.4](0,.35) -- (.25,0) -- (1,.7) -- (.25,.15) -- cycle;} 

\begin{document}

\section{Task 1}
I choose to use Python Programming Language for this class.
	\begin{itemize}
	\item The code was complied using PyCharm;
	\item The compilation resulted with an executable {\bf print}.
	\end{itemize}
\lstinputlisting[language=Python]{Task_1.py}
It has the following output:
\begin{center}
\includegraphics[scale=0.7]{Screenshots/hello.png}\\
{\bf Figure 1.} Code Output.
\end{center}

\section{Task 2}
I have edited my main {\bf README.md} file to include an introduction for the repository. There's now also a table of contents for the homework problems and a link to the software manual I will create.\\
The file can be found by clicking \href{https://github.com/GoByMark/math4610/blob/main/README.md}{here}.



\section{Task 3}
The Taylor series expansion of a function $f(a+h)$ centered at $x = a$ is given by
$$f(a+h) = f(a) + hf'(a) + h^2\dfrac{f''(a)}{2!} + h^3\dfrac{f'''(a)}{3!} + \dots$$
If we don't want to torture yourself, we can approximate it as 
\begin{align*}
f(a+b) &= f(a) + hf'(a)\\
\implies f'(a) & = \dfrac{f(a+h) - f(a)}{h} 
\end{align*} 
Now, to use the centered difference approximation for the first derivative, which is given by 
\begin{align*}
f(a) & = \dfrac{f(a+h) - f(a-h)}{2h}\\
	& = \dfrac{1}{2h}\left[\left(f(a) + f'(a)(h) + \dfrac{1}{2}f''(a)h^2 + \dfrac{1}{6} f'''(\xi_1)(h^3\right) +\left(f(a) + f'(a)(-h) + \dfrac{1}{2}f''(a)(-h)^2 + \dfrac{1}{6} f'''(\xi_1)(-h)^3\right) \right]\\
	& = \dfrac{1}{2h}\left(2f'(a)h + \dfrac{1}{6}\left(f'''(\xi_1) + f'''(\xi_2)\right)h^2\right)\\
	& = f'(a) + \dfrac{1}{12}h^2f'''(\xi_3)
\end{align*}
And we can rearrange to get:
$$\dfrac{f(a+h) - f(a-h)}{2h} - f'(a) = \dfrac{1}{12}h^2f'''(\xi_3)$$
which suggests that the order of accuracy with $h^2$, that's why the centered difference approximation isa second order approximation.
\newpage

\section{Task 4}
The code is the following:
\lstinputlisting[language=Python]{Task_4.py}
With the following output:
\begin{center}
\includegraphics[width=\textwidth]{Screenshots/table.png}\\
{\bf Figure 2.} Code Output.
\end{center}

\vspace{5pt}

\section{Task 5}
There are three finite difference approximations documented \href{https://archive.siam.org/books/ot98/sample/OT98Chapter1.pdf}{here}\footnote{https://archive.siam.org/books/ot98/sample/OT98Chapter1.pdf}.
\begin{itemize}
\item One-sided approximation 
	\begin{enumerate}
	\item[1. ] Forward difference approximation 
		$$\dfrac{f(x+h) - f(x)}{h} = \dfrac{f(x + \delta x) - f(x)}{\delta x}$$
	\item[2. ] Backward difference approximation 
		$$\dfrac{f(x+h) - f(x)}{h} = \dfrac{f(x) - f(x - \delta x)}{\delta x}$$
	\end{enumerate}
\item Centered approximation
	$$\dfrac{f(x + \delta x) - f(x - \delta x)}{2\delta x} - f'(x) = \delta x^2 \dfrac{1}{12} f'''(\xi_3)$$
\end{itemize}
Each of the given example above is found \href{https://www.dam.brown.edu/people/alcyew/handouts/numdiff.pdf}{here}\footnote{https://www.dam.brown.edu/people/alcyew/handouts/numdiff.pdf}.\\
In science and engineering, we often don't know the exact formula for $f(x)$, and given a set of experimental data if we want to know the rate of change of $f(x)$ with respect to $x$, our best choice would be using the finite approximation methods.


\end{document}